\documentclass[letterpaper, 12 pt, conference]{ieeeconf}
\IEEEoverridecommandlockouts
\overrideIEEEmargins
\usepackage{cite}
\usepackage{algorithm}
\usepackage{algpseudocode}
\usepackage{amsmath}
\usepackage{amssymb}
\usepackage{graphics}
\usepackage{graphicx}
\usepackage{multirow}
\usepackage{rotating}
\usepackage{verbatim}
\usepackage[caption=false,font=footnotesize,subrefformat=parens,labelformat=parens]{subfig}
\graphicspath{{graphics/}}
\DeclareGraphicsExtensions{.png}
\title{\bf
Air-Hockey Robot
}

\author{\parbox{5 in}{\centering Daniel Koniar, Matthew Russell, and Steve Thomas}\\
  {\tt\small \{konia013, russe546, thom5159\}@umn.edu}\\
  Department of Computer Science and Engineering\\
  University of Minnesota\\
  Minneapolis, MN 55455\\
}  

\begin{document}
\maketitle
\thispagestyle{empty}
\pagestyle{empty}

%%%%%%%%%%%%%%%%%%%%%%%%%%%%%%%%%%%%%%%%%%%%%%%%%%%%%%%%%%%%%%%%%%%%%%%%%%%%%%%%
\begin{abstract}
This paper evaluates, and attempts to employ, the construction of a Air-Hockey Playing Robotic System in a real-time physical environment.  This project aims to increase understanding of the design, construction, and control of a robotic manipulator through a fast-paced environment…..

I just realized this section should be written last.
\end{abstract}

\section{Introduction}
\label{introduction}
Improvement in both robotic manipulators and computer vision has allowed for the production of complex robotic systems that are capable of reacting to a real-time environment. A variety of specialized robotics, varying from robotic drones \cite{3dr} with ground recognition to self-driving cars \cite{googlecar}, have been constructed using combinations of these components.  

This project combines computer vision and robotic mechanisms in an attempt to design and construct a complex robotic system capable of playing Air Hockey. 

The software that translates images to robotic movements will be the cohesive layer between the computer vision and manipulator hardware. It will provide the necessary computations that will both link the computer vision layer to the robotic manipulator layer and produce a possibility for varying difficulties.  The computer vision part of our project will introduce the sensory component to our system and the robotic mechanism is the response component.

In this project, multiple aspects will be examined and discussed.  One aspect will be the analysis of the actual development and construction of the Air Hockey robot.  Once the robot is in full working operation, the experimental data will be gathered and examined, the metrics of which will be defined in the next paragraph.  If time permitting, some variance will be implemented into this system for further experimental examination and comparison.  The last aspect of the project that will be examined is the entirety of project in a retrospective lens.  The analysis of this will include what could be done to make the system better and what failures were experienced.

There are many key metrics for the Air-Hockey Robot an order to attempt to reach an optimal state. The metrics that will be mainly emphasized are efficiency, reactionary time, accuracy, and repeatability. To further define these, efficiency will be defined as minimizing the movements necessary to be made by the robotic manipulator an order to hit the puck.  Reactionary time will be defined as the amount of time needed to go from raw images to movement input to the robotic manipulator. Accuracy will be defined as how often the puck can be reached and hit in the desired direction [with a small error tolerance].  Repeatability will be defined as the robotic manipulator’s ability to repeat an action precisely, provided the exact same environment/situation. For instance, if a puck were sent from the same exact direction and with the exact velocity one hundred times, repeatability would be the number instances where the robotic manipulator could hit the puck in the same exact way.

\section{Problem Description}
\label{problemdescription}
\subsection{Definition}
\label{definition}
Air Hockey is a two-player game where the players hit a puck across a table using mallets. The table used to play air hockey has elevated borders to prevent the puck from exiting the play field.  The table also has holes in the surface in a regular pattern all across the board through which air is passed through in order to elevate the puck and mallets with a cushion of air, removing surface friction. The goal of air hockey is to hit the puck into the other player’s goal. When the puck enters a player’s goal, the opposing player receives a point.  The first player to reach a predetermined score, or has the highest score after a set amount of time, wins the air hockey match.

An air hockey robotic system will be designed through the combination of computer vision and robotic device manipulation and attempted to be constructed.  The goal of the air-hockey robotic system will be to protect its goal while attempting to defeat the opponent player.

Monolithic-structured software typically acts as the cohesive layering between the robotic and computer vision layers.  This software allows for a translation to be made from the input from one hardware component to output for another hardware component.  This allows for the robotic system to work intelligently in its workspace. 

Through this, it also allows for a goal to defined, or task, and the means to accomplish reaching that goal.  It allows for reactionary response from the robotic mechanism in a changing environment through computational analysis.  One example of this would be automated assembly systems, where computer-vision object-recognition is used in order to have robotic manipulator to grasp and transport said object.

By combining computer vision and robotic mechanisms with a cohesive layer of software, a Air-Hockey playing robot can be implemented with a goal-oriented approach to allow the robotic system to react to a human opponent’s actions and quickly derive a solution. The setup will require an overhead camera which monitors the puck movement and a robotic manipulator with the end effector being the mallet. The manipulator constructed onto the workspace, being the air hockey table, and was interfaced through a computer (Raspberry Pi 2).

The camera will track the puck movement in a two dimensional plane and the data will be processed through the computer vision software to compute the trajectory of the puck.  The trajectory will then be used by the mediary software to predict the movements for the robotic manipulator an order to strike the puck back towards the opposite goal [from which the robotic manipulator is defending].  The mediary software will handle the transition of data from the computer vision software to the robotic manipulator as well as the artificial intelligence schema (which will most likely be a path-finding algorithm with a heuristic) which will compute how the manipulator should react. The end effector will thus be controlled through a combination of robotic mechanisms and artificial intelligence in a reactionary setting.

\subsection{Motivation}
\label{motivation}
The primary motivation of this project is to increase understanding of real-time employments of computer vision in combination with robotic devices. Computer vision is widely used today with its applications ranging from traffic control\cite{aliane} to gesture recognition \cite{bhame}.  More specifically, computer vision is a tool used in robotics to capture and parse images to achieve a predefined goal. The robotic system takes in this image data and performs selective tasks based on the algorithms and functions written within the system. 

The secondary motivation is to experiment with the design and implementation of the software interface that combines two different hardware components. This design of many systems relies on well-defined planning and accommodation for change. This project will increase the understanding of the complexity of systems engineering.

The third motivation is to increase understanding about the significance of robotic accuracy and repeatability. Since the game of air hockey is a fast -paced environment, the robotic system response necessitates a fast reaction time.  This implies that the system’s hardware needs to be precise given proper input.  It also implies that the software needs to properly translate data from images to movement, while account for error.

\subsection{Related Work}
\label{relatedwork}
There has been numerous amounts of research in the design, construction, and control of a complex robotic system capable of playing air hockey in a real-time environment. The designs for this system vary across multiple components and factors.

One approach to the design was to use a robotic arm that had four degrees of freedom for mallet manipulation.  The base of the robotic arm was set just outside of the air hockey playfield while the end-effector grasped the mallet.  This system employed two cameras for puck tracking \cite{namiki}. Another approach had a simplified design compared to that previously mentioned. This approach varied by using a three degrees of freedom robotic arm and only one camera for puck detection \cite{bishop}. The designs from the approaches mentioned avoided complete integration with the air hockey table.

Another approach to the design was to integrate the robotic system into the air-hockey table.  This robotic system traversed the width of the board to align with the puck. The mallet was attached to a spring-loaded mechanism which was released to hit the puck with a high amount of force. This approach limited the workspace for the robotic system, thus limiting complexity.

There has also been research done in the calibration of an air hockey robot. A group in Iran have a system to automatically calibrate the table’s intrinsic parameters for different table setups \cite{alizadeh}.  Additionally, work has been done to consider the interaction between a puck and the paddle hitting it, and what such forces mean for the motion of the puck \cite{ghazvini} \cite{iguchi}. Another group has researched predicting where the puck will be over time and fuzzy logic control involved in this methodology \cite{wang}.

Feature matching is an important part of image processing and will likely be used to track the puck and possibly the robotic manipulator position and velocity. Because motion requires more than one frame, velocity will be determined through the comparison of two image feature matches \cite{rahman}. There are two algorithms to try; SIFT feature matching and thresholded contour detection. SIFT will likely give more accurate results, however, if SIFT feature matching proves too slow for our purposes, there are alternatives involving thresholding and trying to match shapes \cite{wang} \cite{marra} \cite{bishop}. The SIFT feature is decidedly the initial route to take for implementation, but the fallback will either thresholding and shape matching.  The final decision for this will be experimentation, which will possibly present a metric of comparison.


\section{Design}
\label{design}
\subsection{Robotic Manipulator}
\label{design-roboticmanipulator}
The Robotic Manipulator of the Air Hockey playing system is designed to enable movement in two dimensions of motion. This is due to the fact the entire gameplay of an air hockey system is limited to one plane. Therefore, the manipulator for such a system would need inhibited motion on the plane while playing against a player.

The objective of the robotic manipulator is to control the end effector, which in this case is the mallet or striker. This could have been done through number of ways. For one, the use of robotic arm could do job. However, the use of multi-link robotic manipulator could turn inefficient as the control of the such a manipulator would be too complex for such a system. Due to the fast paced nature of the game, it is essential to get rid of any redundancies which could possibly slow down the response time of the system.

Hence, the preferred choice for design of such a system would have to be a two degree of freedom prismatic manipulator. The basic a design consists of mounting the motors on the edges of the workspace to system. The motors are used to drive belts wrapped around them. One set of belts runs along the edges of the air hockey table parallel to each other. These belts control the motion along one direction of the board. Attached onto these belts is the second belt whose length is perpendicular to the them. Attached to its own set of motors, the functionality of the second belt is to control the motion along the second direction of motion. On this belt we attach the mallet, which is the end effector used to play against the player.

The advantage of such a setup enabled for a simpler design which could execute faster in real time and avoid any complexities which would occur with a more sophisticated robotic manipulator. This is a key motivation when dealing with the system as the game demands high speed response from the robotic manipulator.

Besides the design of the robotic manipulator, it is important to select the right hardware needed to build manipulator. For motors the choice varied from steppers, servos and Dc motors. In the end, the choice of two steppers and two Dc’s were taken. The steppers used were the 5VDC 32-Step 1/16 Gearing reduction stepper, which provides a 80 rpm at 12 volts input. The second pair of motors were 12VDC motor. These motors provide great torque at high input voltages. 
The integrated circuits used were the motor driver circuits L293D and the L298N. A total of three integrated circuits were used. Two L293D were used to control the two steppers each and one L298N used drive the pair of DC motors. The L293D and L298N are H-bridges used to interface the motors with a controller. The input voltages applied to the two steppers were 15 volt DC each and 12 volt Dc to Dc motors together. 

\subsection{Computer Vision}
\label{design-computervision}
To facilitate input to the robot, a web camera was utilized as a cheap and easy source of information. The PS3EYE webcam suited this purpose well as it was a cheap component and the Raspberry Pi had drivers for it pre-installed. The computer vision library openCV3 introduced a simple function to grab frames from this camera. 

The puck and striker are colored using base colors, in this case bright green and blue, as these colors are easy to threshold from colors already present in the background of the Air Hockey table which contains primarily reds and blacks and white colors. The image taken from the webcam is reduced in size 4x to improve runtime as the objects the algorithm needs to detect are quite large compared to the background noise. Since our objects are in a fixed plane relative to the camera, the relative size of the image does not matter as long as there is enough resolution to adequately see the colors required. 

To further reduce noise, the a Gaussian blur is applied to the resized image. The image is then converted to HSV to make the thresholding easier, and masked for each color that is needed, using predefined thresholds. These masked images are then passed to OpenCv’s findContours function. The algorithm then finds the largest of the contours above a certain threshold, and computes its centroid.

Because the findContours function returns all possible contours, there is still a lot of noise in the form of tiny contours. This data is filtered to only return the largest contour above some cutoff size as if the size of the contour is sufficiently small the algorithm can be reasonably confidant that the object is not yet in view. In the following example, the puck is colored red and the striker is colored blue, a green dot shows what objects are being tracked by the algorithm.


The algorithm stores the positions it has found previously, and when it has discovered the striker’s position and the puck’s position twice consecutively, it has enough information to plan future trajectories and determine an intercept path for the striker so as to guide it into hitting the puck.

This future point of contact is determined using the equation Px+Pvx*Sy-PyPvy, Syto find the (x, y) coordinate where the puck will cross the striker’s x-axis, where Pis the position of the puck at this timeframe, Pv is the speed of the puck at this time frame, and S is the striker position at this time frame. 



Fig ?: Determining where the puck will cross the striker’s x-axis

\subsection{Robotic System}
\label{design-roboticsystem}
The Robotic System can be divided into three main parts. This division is based on how the Air-Hockey system takes in inputs, processes and responds to the given input, and the generated output. The robotic system was designed with both the robotic manipulator (see section \ref{design-roboticmanipulator}) and the computer vision (see section \ref{design-computervision}) components integrated with a cohesive layer of software. The abstraction of this design is presented in a monolithic structure.  

The cohesive software layer that combines the computer vision and robotic manipulator components of this project handles input and translate it to output for the manipulator.  It would handle communication with the robotic manipulator software and the computer vision software as well as maintaining state data for the system.  This component would integrate the project from being two separate components into one complex system that would be capable of accomplishing a more specialized task.

The Air Hockey playing robot takes in data from the camera mounted on the system.This camera tracks the positions of the puck and mallet simultaneously. The camera captures these position coordinates by grabbing frames from the camera.
These frames are then fed to the Raspberry Pi which processed the information, through the cohesive software layer mentioned previously.

Furthermore, the robotic system would be integrated into the air hockey table. By going with an integrated design, the component of calibration could be lessened. Since there wouldn’t be a necessity to align the robot each time in a three dimensional plane or rather x, y, and z relative to the table position. This was the main reason the design incorporated an integrated design despite numerous works using a detached robotic component (See Section \ref{relatedwork}).

On the processing side, the data fed into the system is firstly filtered and then processed to pinpoint the exact position of the puck on the workspace. Based on these positions the (NOT DONE)

\section{Results}
\label{results}
\subsection{Mallet Manipulation}
\label{results-malletmanipulation}
TODO

\subsection{Puck Detection}
\label{results-puckdetection}
Two methods were tried to detect moving objects in the web camera input, Scale Invariant Feature Transform(SIFT) feature detection and thresholded contour blob detection. SIFT was chosen initially for its ability to pick out scale invariant features, and seemed like it would be a good fit for the project. In our tests however, basic SIFT feature detection was rather slow and error prone, even in C++ introducing noticeable lag in the output stream. In addition it had quite a lot of trouble detecting shapes in a very noisy environment and was deemed unsuitable for the purposes required. 


Thresholded contour mapping was tried next, as the puck and striker really only needed to be defined by their position, orientation did not matter. In addition, both objects were large and colored brightly, which made identifying thresholds to mask them from the rest of the background quite easy. 

The algorithm thresholds on these colors using values determined beforehand, and this thresholded image is passed into openCV3’s findContours function. Then the largest of these found contours is the result. This method was quite a lot faster than the SIFT algorithm, and did a remarkable job at tracking moving objects in real time. To further speed up this algorithm the input image stream was reduced in size by 4x, improving the runtime even more while still producing adequate results. 


Fig ?: Threshold masking for blue and green for Fig (ABOVE) using thresholding and openCV3’s built in findContours function.

\subsection{Air-Hockey Playing Robotic System}
\label{failure}

\section{Analysis}
\label{analysis}
\subsection{Mallet Manipulation}
\label{analysis-malletmanipulation}
The speeds achieved from the DC motors used to control the end effector (the air hockey mallet) were ideal.  However, due to the lack of precise control over the DC motors made the end effector difficult to control accurately as the opponent’s goal was approached.  Since the guide rails mounted to the frame that encased the Air Hockey table weren’t precise and the rails too rough, the robotic manipulator experienced irregular friction as the length of the table was traversed. Compiling on to this issue, the DC motors were also underpowered through the H-bridge Integrated Circuits used, creating a bad trade-off. This trade off was increased control through the H-bridge and the cost of significantly decreased speed and torque. This allowed for the software to control the DC motors to rotate about z-axis in both directions, but at the cost completely halting when attempting to move the robotic manipulator in both negative and positive y-directions.

The speeds achieved by the stepper motors were slower.  The trade off for speed was the precision of movement in the x-direction was more precise and in terms of number of steps. This choice was costly since the speed generated by the stepper motors were far too slow to be capable of functioning in a fast-paced real-time environment.

The wiring for the robotic manipulator was done through a breadboard. This was not the best of options for circuitry since it produced a costly tradeoff.  This tradeoff was simplicity for decreased voltage.  Since a breadboard is designed for lower powered circuits, this limited our options in motors since the ones that could be used had to consume less power. 

The H-bridges that were used weren't designed to give enough power to the motors which made it difficult to rotate the motors with enough effective torque and speed.  Torque was needed to overcome the friction from the end effector on the board and the linear bearing on the side-guide rails.  Speed was needed to react in the real-time environment.  Since steppers and dc motors were being used, in order to control voltage input to them, H-bridges had to be used. This caused an uncompromising situation where no trade-off could suffice to create functional operation for the robotic system in regards to the goal of playing air hockey. This will be further explained in Section \ref{analysis-airhockeyplayingroboticsystem}.

\subsection{Puck Detection}
\label{analysis-malletmanipulation}
The SIFT feature detection algorithm was so slow likely because detecting features is a very complex task better suited to still frame images rather than a video feed. Thresholded contour detection presented a faster alternative because it consisted mainly of array manipulations which the Python numpy library is exceedingly good at \cite{numpy}. Due to this, the thresholding operations for contour detection can be optimized in C++ to provide faster results.

The thresholded contour detection also meant picking out colored objects is an inherent operation as part of the thresholding process, rather than an additional process that would be required after applying the SIFT algorithm. This is likely why the findContours function is so fast, the SIFT algorithm just has to consider too much detail.

\subsection{Air-Hockey Playing Robotic System}
\label{analysis-airhockeyplayingroboticsystem}

\section{Conclusion}
\label{conclusion}
The failure to construct an air hockey robotic system was mostly due to poor design imposed by resource constraints and a misunderstanding of just how much work this project would entail. The constant state of improvisation in the construction of the robotic system and the deterioration of the initial design led to an incomplete and non-functional robotic system in terms of achieving the initial goal put forth. Poor choice in motors and frame design only exacerbated this problem.

Despite failure of this system, important information was gathered through this experiment. First, the importance of planning and the ability to adapt to change is a key component to designing any complex system. Coordination needs to be maintained across all components and hardware when developing complex robotics like this, and having a well thought out plan with designs for every part and how they fit together are incredibly important. 

SIFT feature matching was determined to be too slow and detailed for the problem being solved. Thresholded contour detection worked much faster without providing needless data and allowed for detection based on color as part of the process, allowing for a very efficient algorithm. The output is not always consistent as noise can interfere greatly but with a well tuned set of thresholds the chance of this happening in a well lit environment can be minimized quite effectively.

Were this project to be repeated, designing the robot using 3D modelling software would be key, allowing the designer to determine what parts need to be bought, and what parts need to be custom made. In addition, determining a proper circuit diagram to power all the motors with enough power to operate quickly and smoothly without overheating any components is also vitally important. the kinematics of both the puck and the striker should be worked out beforehand to make programming the control logic easier. Starting the project without these detailed plans is only liable to not work and drastically exceed budget.

\section{Future Work}
\label{futurework}
If more time was given to correct the robotic system built and experimentation, many changes would have to be made.  One of these changes would be the transition from a breadboard circuitry to a soldered circuit with higher gauge wiring and higher amperage-tolerating components. If this were done, it would permit more power to run to the motors, which would allow for more powerful motors to be used. Then the current motors that were used in this project’s setup would be changed to ones that provide much more torque and speed.  Also, the stepper motors used could be upgraded to a more powerful stepper since the higher power DC motors would be able to handle the movement.

Smoother guide rails would greatly improve motor performance, but are not cheap. Consider getting thinner rods if this is required as the rods used in this project were easily sturdy enough to not flex at all from the weight of the rest of the system.

The components holding the hardware together and to the frame should have been modelled in a 3D Computer Aided Design program and 3D printed, instead of cobbled together from spare parts and wood. This change would offer greater precision and stability in the design and would be less likely to break or become misaligned during testing and usage. This step requires knowing what hardware the project will require beforehand, and the dimensions of said.

The algorithm currently does not take into account the striker’s speed as part of its calculations because there was never enough time to empirically test how fast it could move in any direction. If this parameter could be determined, it would make determining an optimal intercept point in two dimensions possible using a line intersection algorithm. 

Additionally, detecting the edges of the air hockey table could be computed as an additional thresholding step, if the edges are colored with some detectable and distinct color. With this new information, it should be simple to plan the puck’s motion though bounces and rebounds, and define limits on the available movement of the striker. A different algorithm would be required to detect these borders however, as the current implementation finds the center of the largest contour.

Planning motion of the puck after the striker meets the it could also be calculated, which would let the algorithm determine the best angle of intercept to knock the puck back at the opponent’s goal. This final additional step would provide a much more engaging and challenging opponent.

%%%%%%%%%%%%%%%%%%%%%%%%%%%%%%%%%%%%%%%%%%%%%%%%%%%%%%%%%%%%%%%%%%%%%%%%%%%%%%%%
\bibliographystyle{IEEEtran}
\bibliography{IEEEabrv,AirHockeyRobot}
\end{document}
