\documentclass[letterpaper, 10 pt, conference]{ieeeconf}
\IEEEoverridecommandlockouts
\overrideIEEEmargins
\usepackage{cite}
\usepackage{algorithm}
\usepackage{algpseudocode}
\usepackage{amsmath}
\usepackage{amssymb}
\usepackage{graphics}
\usepackage{graphicx}
\usepackage{multirow}
\usepackage{rotating}
\usepackage{verbatim}
\usepackage[caption=false,font=footnotesize,subrefformat=parens,labelformat=parens]{subfig}

\title{\bf
A New Approach for Object Recognition Engine in Robotics
}

\author{\parbox{5 in}{\centering Susan Jones and Robert Smith}\\
  {\tt\small \{sjones, rsmith\}@cs.umn.edu}\\
  Department of Computer Science and Engineering\\
  University of Minnesota\\
  Minneapolis, MN 55455\\
}  

\begin{document}
\maketitle
\thispagestyle{empty}
\pagestyle{empty}

%%%%%%%%%%%%%%%%%%%%%%%%%%%%%%%%%%%%%%%%%%%%%%%%%%%%%%%%%%%%%%%%%%%%%%%%%%%%%%%%
\begin{abstract}
\end{abstract}

%%%%%%%%%%%%%%%%%%%%%%%%%%%%%%%%%%%%%%%%%%%%%%%%%%%%%%%%%%%%%%%%%%%%%%%%%%%%%%%%
\section{Introduction}
3D point cloud data provided by low-cost RGB-D sensors \cite{kinect}, has
allowed for the collection of feature rich datasets. These widely available
sensors provide synchronized color and depth images and are increasingly being
used in robotic applications such as object detection and classification.
Although these datasets have opened new avenues of research, it remains
undetermined as to which feature descriptors and machine learning classifiers
provide optimal object recognition performance. 

The remainder of this paper is organized as follows. Related work is mentioned 
in Section \ref{sec:related_work}. In Section \ref{sec:problem_statement}, we
state the problem and provide our solution. Experimental results are presented
and discussed in \ref{sec:experimental_results}. We conclude in 
Section \ref{sec:conclusion} with an outlook of ongoing and future work.

%%%%%%%%%%%%%%%%%%%%%%%%%%%%%%%%%%%%%%%%%%%%%%%%%%%%%%%%%%%%%%%%%%%%%%%%%%%%%%%%
\section{Related Work}
\label{sec:related_work}

%%%%%%%%%%%%%%%%%%%%%%%%%%%%%%%%%%%%%%%%%%%%%%%%%%%%%%%%%%%%%%%%%%%%%%%%%%%%%%%%
\section{Problem Statement}
\label{sec:problem_statement}

%%%%%%%%%%%%%%%%%%%%%%%%%%%%%%%%%%%%%%%%%%%%%%%%%%%%%%%%%%%%%%%%%%%%%%%%%%%%%%%%
\section{Experimental Results}
\label{sec:experimental_results}

%%%%%%%%%%%%%%%%%%%%%%%%%%%%%%%%%%%%%%%%%%%%%%%%%%%%%%%%%%%%%%%%%%%%%%%%%%%%%%%%
\section{Conclusion and Future Outlook}
\label{sec:conclusion}

%%%%%%%%%%%%%%%%%%%%%%%%%%%%%%%%%%%%%%%%%%%%%%%%%%%%%%%%%%%%%%%%%%%%%%%%%%%%%%%%
\bibliographystyle{IEEEtran}
\bibliography{IEEEabrv,project}
\end{document}
